% /solutions/conference-talks/conference-ornate-20min.fr.tex, 22/02/2006 De Sousa

\documentclass{beamer}




\mode<presentation> {
  \usetheme{Warsaw}
  % ou autre ...

  \setbeamercovered{transparent}
}


\usepackage[french]{babel}
% or autre comme par exemple \usepackage[english]{babel}

\usepackage[latin1]{inputenc}
% or autre

\usepackage{times}
\usepackage[T1]{fontenc}


\title[Titre court] 
{MY TWIN}

\subtitle {}

\author[] 
{Developers: \\Hernan Ullon\inst{1} \and Kevin Calderon\inst{2}  \and Cesar San Lucas\inst{3}}



\subject{La théorie de l'Informatique}


% \pgfdeclareimage[height=0.5cm]{le-logo}{université-logo-nomfichier}
% \logo{\pgfuseimage{le-logo}}



\AtBeginSubsection[] {
  \begin{frame}<beamer>{DESCRIPCION}
    \tableofcontents[currentsection,currentsubsection]
  \end{frame}
}


\begin{document}

\begin{frame}
  \titlepage
\end{frame}

\begin{frame}{DESCRIPCION}
  \tableofcontents
  % Vous pouvez, si vous le souhaiter ajouter l'option [pausesections]
\end{frame}


\section{IDEA}

\subsection{Problematica}

\begin{frame}{Problematica}{Ideas:\\Simples\\Dificiles\\Sencillas\\\\Ninguna Convencia}


%IMAGENES


\end{frame}

\subsection{Idea Final}

\begin{frame}{Idea Final}

%IMAGENES

\end{frame}




\section{MY TWIN}

\subsection{Descripcion}

\begin{frame}{Descripcion}
\begin{itemize}
\item
    Divertida\pause
    \item
     Realista\pause
    \item
    Multimedia

    \end{itemize}
\end{frame}



\subsection{Recursos}

\begin{frame}{Multimedia}

\end{frame}



\section*{Résumé}

\begin{frame}{Résumé}

  % Faîtes un résumé *très court*.
  \begin{itemize}
  \item
    Le \alert{premier message principal} de l'exposé en une ligne ou
    deux.
  \item
    Le \alert{deuxième message principal} de l'exposé en une ligne ou
    deux.
  \item
    Peut-être un \alert{troisième message}, mais pas plus que ça.
  \end{itemize}

  % Les perspectives suivantes sont facultatives.
  \vskip0pt plus.5fill
  \begin{itemize}
  \item
    Perspectives
    \begin{itemize}
    \item
      Problème non résolu.
    \item
      Autre problème non résolu.
    \end{itemize}
  \end{itemize}
\end{frame}



\appendix
\section<presentation>*{\appendixname}
\subsection<presentation>*{Lectures complémentaires}

\begin{frame}[allowframebreaks]
  \frametitle<presentation>{Lectures complémentaires}

  \begin{thebibliography}{10}

  \beamertemplatebookbibitems

  \bibitem{Auteur1990}
    A.~Auteur.
    \newblock {\em Livre de quelque chose}.
    \newblock Une Edition, 1990.


  \beamertemplatearticlebibitems

  \bibitem{Quelqu'un2000}
    S.~Quelqu'un.
    \newblock Et ceci et cela.
    \newblock {\em Journal de ceci et cela}, 2(1):50--100,
    2000.
  \end{thebibliography}
\end{frame}

\end{document}